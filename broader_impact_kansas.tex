%\section{Broader Impacts}


\subsection{University of Kansas}
\label{BI}
\noindent \textbf{A High School Research Program:} Rudnick's broader
impact provides for the continuation of a successful research-based
outreach component at Lawrence High School (LHS; see
\S\ref{Sec:prior_rudnick}). This program is timely, as the Kansas
school system is an exemplar of the nationwide debate on the role of
science in the classroom. Crucial to increasing and clarifying the
role of science is educating students and training teachers.
%% What has become clear after the first year of this program is the
%% great value in bringing expert instructional support into the
%% classroom, as the partnering high school teacher lacks the research
%% experience at this point to adequately lead the program. The KU
%% employee (an ex-student) who is carrying out most of the in-class
%% instruction is proving crucial in this role. 
This program is currently funded through the end of the 2017-2018
academic year. The purpose of the current proposal is to: 1) continue
the funding of a KU student to work in the classroom, 2) develop the
project to the point of sustainability by training the high school
teacher in research-based teaching methods, 3) develop a peer
instruction model to expand the program to 20 students without
additional personnel costs, and 3) expand the program to 20
students. The school district fully supports this program (see letters
from McEwen and Bricker.)

\noindent \textbf{Implementation:} LHS has a 60\% higher percentage of
African Americans and a 340\% higher percentage of Native Americans,
when compared to other Kansas High Schools. Rudnick will continue his
successful recruitment of female students and those from
underrepresented minorities. The current proposal will extend the
ongoing project to measure SFRs of galaxies in the local filaments.
%% Coupled with student measurements
%% being made of EDisCS clusters at $0.4<z<0.8$, this will allow the
%% students to constrain the evolution in the SFRs of cluster galaxies
%% from $0.4<z<1.5$.
The requested funding will be used to hire a physics or astronomy
graduate student interested in education or an employee with similar
qualifications. This employee will be the main contact person in the
classroom and will lead the day-to-day instruction and supervision of
the learning teams during the execution of the project. Rudnick's
main tasks will be to coordinate the program, decide on the exact
curriculum based on our assessment process (see below), attend the
class once every week, and ensure that the program becomes sustainable
in future years. Through Rudnick's \textit{continuous} support and
heavy involvement in the program, the high school teacher is able to devote more of his time to training students to aid in peer instruction, which will allow us to expand in future years without additional personnel costs.

%% The KU graduate program has a solid record of students with an
%% education background or with an interest in education and outreach.
%% Rudnick currently has an ex-student employee who is excelling in his
%% role, but if needs be there will no problem recruiting a new student
%% with the necessary skills and motivation.

\noindent
\textbf{Assessment:} %% The program has been constructed according to the
%% theory of backwards design, in which the broad goals are defined
%% first, then specific goals, followed by the means of assessment, and
%% finally the details of implementation. The goals and implementation
%% are outlined in the previous paragraphs. 
The assessment consists of an end of project presentation and paper
for each student. Their presentations are made to KU faculty during a
mini-conference at KU. All students are given the pre- and post-course
Light and Spectra Concept Inventory \citep{bardar06}.  We also make
students give multiple oral presentations throughout the semester and
have a rubric to evaluate their improvement over the course of the
project.
%% One ongoing task of the employee will be to research and develop
%% new assessment tools and to modify the program accordingly to the
%% results. As an example, in the Fall of 2014, Rudnick has decided
%% to focus more
%on student presentations in the astronomy instruction part of the
%course, to give the students practice in actively seeking out and
%presenting their own astronomy knowledge.
This project satisfies important elements of several of the Kansas
state science standards, i.e.~``Science as Inquiry'' via the process
of research and of communicating their results, benchmark 2 and 3 of
the Physics standards via learning about the electromagnetic spectrum
and how it relates to astronomical phenomena, benchmark 4 of the earth
and space sciences standards relating to general astronomy, and
benchmark 2 of the ``history and nature of science'' via the
understanding gained of the scientific process.
