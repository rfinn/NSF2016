\documentclass[12pt, preprint]{aastex}

\usepackage{epsfig, xcolor, footnote, mdwlist, wrapfig}
\usepackage{hyperref}
\usepackage[flushleft]{threeparttable}
\usepackage{sidecap}
\usepackage{simplemargins}
\usepackage[compact]{titlesec}
\usepackage{times}
\usepackage{amsmath}
\usepackage[]{natbib}
\usepackage{sidecap}

%\usepackage[font=footnotesize]{caption}
%\usepackage{caption}
%\usepackage[font=footnotesize, figurename=Figure, labelsep=endash]{caption}
%\usepackage{multirow}

\setlength\parindent{0pt}
\setleftmargin{1.05in}
\setrightmargin{1.05in}
\settopmargin{1in}
\setbottommargin{1.2in}

%\setlength{\oddsidemargin}{-0.2 in}
%\setlength{\evensidemargin}{-0.2 in}

\pagenumbering{arabic} %roman/Roman/arabic
\pagestyle{plain}

\renewcommand{\baselinestretch}{0.94} 
\newcommand{\isedfit}{iSEDfit}
\newcommand{\redmapper}{redMaPPer}

\begin{document}

{\Large{\bf Facilities, Equipment, and Other Resources}}
\vspace*{2mm}

%{\em This section of the proposal is used to assess the adequacy of the
%  resources available to perform the effort proposed to satisfy both the
%  Intellectual Merit and Broader Impacts review criteria. Proposers should
%  describe only those resources that are directly applicable. Proposers should
%  include an aggregated description of the internal and external resources (both
%  physical and personnel) that the organization and its collaborators will
%  provide to the project, should it be funded. Such information must be provided
%  in this section, in lieu of other parts of the proposal (e.g., budget
%  justification, project description). The description should be narrative in
%  nature and must not include any quantifiable financial information. Reviewers
%  will evaluate the information during the merit review process and the
%  cognizant NSF Program Officer will review it for programmatic and technical
%  sufficiency.
%
%  Instructions: Upload an aggregated description of the internal and external
%  resources (both physical and personnel) that the organization and its
%  collaborators will provide to the project, should it be funded. Describe only
%  those resources that are directly applicable. The description should be
%  narrative in nature and must not include any quantifiable financial
%  information. If there are no Facilities, Equipment, or Other Resources
%  identified, a statement to that effect should be indicated in this section and
%  uploaded into FastLane. See GPG II.C.2.i for more information.}

Siena College provides all the structural scientific and research facilities
necessary to carry out the proposed research program.  The PI has dedicated
office space in Roger Bacon Hall, which houses the Department of Physics \&
Astronomy within the School of Science.  He and his students utilize a
$600$~square foot laboratory dedicated for astrophysics research.  This lab
currently holds four dedicated, reasonably modern Dell computers running Redhat
Linux, providing remote access to a departmental server housed in the College's
Data Center.  

%The PI also has a computer server with several terrabytes of free
%storage and sufficient memory and processing power for him and his students to
%conduct research; the server was purchased with the PI's startup funds.  

The PI has assembled a well-qualified group of collaborators who will contribute
the additional skills and resources necessary for the project to be successful:
\vspace*{-2mm}
\begin{itemize}
\item{Collaborators Eli Rykoff and Eduardo Rozo are the developers of the
  \redmapper{} algorithm.  They have committed to running \redmapper{} on the
  DECaLS, MzLS, and BASS survey data after each data release and to delivering
  the resulting sample of central galaxies and richness estimates to the PI and
  his students.}
%  They will also assist with the stellar mass function analysis,
%  which will require a proper implementation of the probabalistic aspects of the
%  \redmapper{} catalog, and with the image stacking.}
\item{Collaborator Dustin Lang is a core member of the DECaLS, MzLS, and BASS
  survey teams, and is the principle developer of the Tractor.  The PI and his
  students will work closely with Lang to implement all the custom,
  science-critical enhancements to the DECaLS/Tractor pipeline.}
\item{Collaborators David Schlegel and Arjun Dey are the Co-Is of DECaLS, Dey is
  PI of the MzLS survey, and both collaborators are actively engaged in the
  design and execution of the BASS survey.  They have committed to having all
  the value-added data and catalogs produced as part of this research program
  archived and served to the public through the NOAO Science Archive and/or the
  Berkeley/NERSC Science Gateways portal, with documentation to be delivered by
  the PI and his students.  In addition, both collaborators are supportive of
  having undergraduate students participate in the DECaLS and MzLS observing
  runs.}
\item{Collaborator Brian Nord has extensive experience with a broad array of
  education and public outreach (E/PO) efforts, including spearheading E/PO for
  the Dark Energy Survey (DES), and will provide feedback on the development of
  the {\em Broader Impacts} aspects of this proposal.}
%  .  He will assist with the development and
%  implementation of the {\em Broader Impacts} aspects of this proposal, and has
%  committed to helping supervise the undergraduate students engaged with this
%  part of the proposal.}
\end{itemize}

%Siena College provides a unique array of outstanding scientific, reference, and
%research facilities for a small liberal arts college.  The J. Spencer and
%Patricia Standish Library is a 72,000 square foot building that provides
%wireless access, over 100 computer workstations, a computer lab and a 40-seat
%screening room.  Moreover, through collaborative exchange agreements with nearby
%Rensselaer Polytechnic Institute (RPI) and the University of Albany, there is
%extensive access to first-rate research library facilities.
%
%The Morrell Science Center (MSC) is a 55,000 square foot science center with 24
%research labs, 10 teaching labs and three support areas on three floors. There
%is also a small machine shop for manufacture of prototype parts and lab
%apparatus.  The School of Science maintains an independent network that includes
%two Apache web servers integrated with two file servers (Dell PowerEdge 2650+
%servers with Dual Xeon processors running RedHat Enterprise Linux ES 3.0), and
%an Oracle-based database server.  Collectively, this amounts to over 10TB of
%total storage. The school also maintains twelve electronically enhanced
%classrooms (EECs) and labs that each includes a projector, sound system, and
%podium computer.  The Department of Physics \& Astronomy uses technology
%extensively for teaching, with both Python and MATLAB as the standard program in
%the majors sequence.  LabView and Mathematica are also used.
%
%Siena has recently (Summer 2014) completed the construction of the Stewarts
%Advanced Instrumentation \& Technology (SAInT) Center with the goal of
%establishing Siena as a leader in undergraduate education in scientific
%instrumental resources and training. The Center is located in a newly renovated
%laboratory space in MSC. New Instrumentation was purchased including multiple
%mass spectrometers, an atomic force microscope, a scanning electron microscope,
%and other analytic equipment.  The college has hired Dr. Kristopher Kolonko to
%act as the director of the SAInT Center to maintain the instrumentation, provide
%training and support for students and faculty, and coordinate of the use of
%instrumentation with both internal and external users.
%
%Another new addition (2013) is the Siena College High Performance Computing
%Center (HPCC).  The HPCC cluster has 252 2.3 GHz Intel Xeon (E5-2630) cores and
%20.5 TB of global storage.  Each worker node has 500 GB of local storage and 32
%GB of RAM. A full suite of software tools and compilers are available on the
%cluster and both the PI and students have access to this resource.
%
%Over the past decade, Siena has undergone a transformation as it relates to
%sponsored research activities and undergraduate research participation due to
%the exponential growth in grant monies procured from federal funding agencies
%such as NSF, NASA, NEH and the U.S. Department of Education.  The College
%established a Center for Undergraduate Research and Creative Activities (CURCA)
%to foster a campus-wide culture in which all undergraduates are engaged in
%investigations conducted in collaboration with a faculty mentor that makes an
%original intellectual or creative contribution to a discipline or the
%community. The center's resources facilitate and enhance the research experience
%for all undergraduates, including students funded by external grants. Students
%are encouraged to take part in national poster events such as the Council on
%Undergraduate Research ``Posters on the Hill" event held each year at the
%Capitol in Washington, DC, and the National Conference for Undergraduate
%Research event that is attended by more than 200 students and faculty from some
%of the best colleges and universities in the nation. The ultimate goal of the
%center is to prepare highly engaged and motivated students to pursue
%post-baccalaureate opportunities upon graduation at top-tier national
%universities.  

\end{document}
