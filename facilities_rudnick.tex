%%%%%%%%%%%%%%%%%%%%%%%%%%%%%%%%%%%%%%%%%%%%%%%%%%%%%%%%%%%%%%%%%%%%%%%%%%%%%%%%
\documentclass[11pt,preprint]{aastex}

\usepackage{graphicx}
\usepackage[vmargin=1in,hmargin=1in]{geometry}
\usepackage{natbib}
\usepackage{epsfig}
\setlength{\parindent}{0in}

\begin{document}

%%%%%%%%%%%%%%%%%%%%%%%%%%%%%%%%%%%%%%%%%%%%%%%%%%%

\noindent
{\Large\bf Facilities, equipment, and other resources}\\
{\Large Gregory Rudnick - University of Kansas}


{\bf Office Space:} The offices for co-PI Rudnick, graduate student,
and hourly worker will be located in the Physics \& Astronomy
department at the University of Kansas.

{\bf Computer support:} The Physics \& Astronomy department provides
computer support including a fast internet connection for all faculty
and graduate students.  There are two full time departmental system
administrators who maintain the computers and internet connections.
The department has available printers.  The research graduate student
will have a desktop funded by this proposal.   The University of Kansas has created a file storage service titled ?Research File Storage?.  The service provides nightly backup and off-site storage of backups for disaster recovery.  It is also scalable to serve both small and large storage needs.  The service is monitored 24/7 by experienced KU Information Technology Staff, requires authentication for identity management purposes, and uses Active Directory to manage authorization.  Finally, the storage costs are heavily subsidized by the university and a nominal fee is charged and calculated by the university rate committee in accordance with A-21.  All data for the project will be stored on RFS, which also has a very fast internet connection to Mallott Hall, where co-PI Rudnick and his graduate student are located.

{\bf Software:} Nearly all analysis will be done using publicly
available software.  An IDL license will be purchased for the graduate
student and PI for additional analysis tasks.

{\bf Phillips Claude Telescope:}  The University of Kanas is a 15\% partner in the PCT, which is located at Mt. Laguna Observatory (MLO).  The PCT is a 1.25m reflecting telescope with a 4k$\times$4k CCD imager that has a 22$\times$22 arc minute field of view.  The telescope is remotely controlled from an observing station at the university of Kansas with someone on site at (MLO) for emergencies and to monitor the local weather.  All observations will be carried out at the University of Kansas.  The 15\% share is granted on a semester basis and so can be given to partners at times optimal for their science program.  


{\bf Instructional Space:} The high school outreach program that will
be conducted by Rudnick will be carried in the classroom of Andrew
Bricker, a physics and astronomy teacher at Lawrence High School.  The
Lawrence School District has approved this allocation (see supporting
letters from Bricker and McEwen).






\end{document}
%%%%%%%%%%%%%%%%%%%%%%%%%%%%%%%%%%%%%%%%%%%%%%%%%%%%%%%%%%%%%%%%%%%%%%%%%%%%%%%
%%%%%%%%%%%%%%%%%%%%%%%%%%%%%%%%%%%%%%%%%%%%%%%%%%%%%%%%%%%%%%%%%%%%%%%%%%%%%%%
