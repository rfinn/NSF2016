
\documentclass[11pt]{article}

\usepackage[top=1in, bottom=1in, left=1in, right=1in]{geometry}
%\addtolength{\textheight}{48mm}
%\addtolength{\textwidth}{48mm}
%\addtolength{\voffset}{-24mm}
%\addtolength{\hoffset}{-24mm}

\usepackage{times}
%\usepackage{amsmath}
%\usepackage{amssymb}
%\usepackage{indentfirst}

\newcommand{\HRule}{\rule{\linewidth}{0.7mm}}
%\pagestyle{myheadings}
\markright{ }

\begin{document}

%\setcounter{page}{3}
%\renewcommand{\thepage}{E--\arabic{page}}
\renewcommand{\thepage}{}

\begin{center}
{\Large Biographical Sketch:} \\
\vspace{2.5mm}
{\LARGE \bf Gregory H. Rudnick}
\end{center}
\vspace{1.0mm}

\begin{flushleft}
\vspace{-0.3in}
{\large {\bf \textsc{Professional Preparation}}
\hrulefill} \\
\end{flushleft}

\vspace{-0.1in}
\begin{tabular}{l @{\hspace{0.6cm}} l @{\hspace{0.6cm}} l @{\hspace{0.8cm}} r}
University of Illinois, IL    & Physics         & B.S. & 1996 \\
University of Arizona, AZ      & Astronomy       & Ph.D. & 2001 \\
Max-Planck-Institute for Astrophysics & Astronomy, D & Postdoc & 2001 - 2004 \\
National Optical Astronomy Observatory & Astronomy, D & Leo Goldberg Fellow & 2004 - 2008
\end{tabular}

\begin{flushleft}
\vspace{-0.1in}
{\large {\bf \textsc{Appointments}}
\hrulefill}
\end{flushleft}

\vspace{-0.1in}
\indent Associate Professor of Astronomy, University of Kansas (August 2013 - present) 

\indent Assistant Professor of Astronomy, University of Kansas (August 2008 - July 2013) 


\begin{flushleft}
\vspace{-0.1in}
{\large {\bf \textsc{Related Products}}
\hrulefill}
\end{flushleft}

\vspace{-0.07in}
\hangindent=1.5cm \hangafter=1 
{\it Disc colors in field and cluster spiral galaxies at $0.5 < z  < 0.8$}, {{Cantale}, N. and {Jablonka}, P. and {Courbin}, F., {\bf {Rudnick}, G., }
	{Zaritsky}, D., {Meylan}, G., {Desai}, V., {De Lucia}, G. 
	{Arag{\'o}n-Salamanca}, A., {Poggianti}, B.~M., {Finn}, R., and 
	{Simard}, L.}, 2016, A\&A, 589, A82

\hangindent=1.5cm \hangafter=1 
{\it A Tale of Dwarfs and Giants: Using a $z=1.62$ Cluster to
  Understand How the Red Sequence Grew Over the Last 9.5 Billion
  years}, {{\bf {Rudnick}, G.}, {Tran}, K.-V., {Papovich}, C.,
  {Momcheva}, I., and {Willmer}, C.}, 2012, ApJ, 755, article id. 14

\hangindent=1.5cm \hangafter=1 
{\it Dust Obscured Star Formation in Intermediate Redshift Clusters}, Finn, R., , Desai, V., {\bf
  Rudnick, G.}, Poggianti, B., Bell, E., and 6 co-authors, 2010, ApJ, 720, 87

\hangindent=1.5cm \hangafter=1 
{\it A Spitzer-selected Galaxy Cluster at $z = 1.62$}, Papovich, C.,
Momcheva, I., Willmer, C.~N.~A., Finkelstein, K.~D., Finkelstein,
S.~L., Tran, K.-V., Brodwin, M., Dunlop, J.~S., Farrah, D., Khan,
S.~A., Lotz, J., McCarthy, P., McLure, R.~J., Rieke, M., {\bf Rudnick, G.},
Sivanandam, S., Pacaud, F., \& Pierre, M.\ 2010, ApJ, 716, 1503-1513

\hangindent=1.5cm \hangafter=1 
{\it The Rest-frame Optical Luminosity Function of Cluster Galaxies at
  $z < 0.8$ and the Assembly of the Cluster Red Sequence}, {\bf Rudnick,
  G.}, {von der Linden}, A., {Pell{\'o}}, R.,
{Arag{\'o}n-Salamanca}, A., and 11 co-authors, 2009, ApJ, 700, 1559

\begin{flushleft}
\vspace{-0.1in}
{\large {\bf \textsc{Other Significant Products}}
\hrulefill}
\end{flushleft}

\vspace{-0.1in}
\hangindent=1.5cm \hangafter=1 
{\it What are the Progenitors of Compact, Massive, Quiescent Galaxies
  at z = 2.3? The Population of Massive Galaxies at $z > 3$ from NMBS
  and CANDELS}, Stefanon, M., Marchesini, D., {\bf Rudnick, G.~H.}, Brammer,
G.~B., \& Whitaker, K.~E., 2013, ApJ, 768, 92
%\vspace{0.15cm}

\hangindent=1.5cm \hangafter=1 {\it The Number Density and Mass
  Density of Star-forming and Quiescent Galaxies at 0.4$<$z$<$2.2},
Brammer, G.~B., Whitaker, K.~E., van Dokkum, P.~G., Marchesini, D.,
Franx, M., Kriek, M., Labb{\'e}, I., Lee, K.-S., Muzzin, A., Quadri,
R.~F., \textbf {Rudnick, G.}, and Williams, R.\ 2011 \ ApJ.\ 739, 24

\hangindent=1.5cm \hangafter=1 {\it The Rise of Massive Red Galaxies:
  The Color-Magnitude and Color-Stellar Mass Diagrams for $z_{phot} <
  2$ from the Multiwavelength Survey by Yale-Chile}, Taylor, E.~N.,
Franx, M., van Dokkum, P.~G., Bell, E.~F., Brammer, G.~B., {\bf
  Rudnick, G.}, Wuyts, S., Gawiser, E., Lira, P., Urry, C.~M., \& Rix,
H.-W.\ 2009, ApJ, 694, 1171-1199


%% {\it The Lopsidedness of Present-Day Galaxies: Connections to the
%%    Formation of Stars, the Chemical Evolution of Galaxies, and the
%%    Growth of Black Holes}, Reichard, T. A., Heckman, T. M., {\bf
%%      Rudnick, G.}, Brinchmann, J., Kauffmann, G., Wild, V., 2009, ApJ,
%%    691, 1005
%\vspace{0.15cm}

\hangindent=1.5cm \hangafter=1 
{\it Measuring the Average Evolution of Luminous Galaxies at
  $z<3$: The Rest-Frame Optical Luminosity Density, Spectral Energy
  Distribution, and Stellar Mass Density}, {{\bf Rudnick, G.},,
  {Labb{\'e}}, I., {F{\"o}rster Schreiber}, N.~M., {Wuyts}, S.,
  and 10 coauthors, 2006, ApJ, 650, 624
%\vspace{0.15cm}

\hangindent=1.5cm \hangafter=1 
{\it The Rest-Frame Optical Luminosity Density, Color, and
  Stellar Mass Density of the Universe from $z=0$ to $z=3$}, {\bf
  Rudnick, G.}, Rix, H.-W., Franx, M., and 11 coauthors, 2003, ApJ, {\bf 599}, 847-864.

\begin{flushleft}
\vspace{1.5in}
{\large {\bf \textsc{Synergistic Activities}}
\hrulefill}
\end{flushleft}

\vspace{-0.1in}

\indent Fall 2015 -- Spring 2016: Chaired working group for LSST/NOAO/Kavli report "Maximizing Science in the Era of LSST: A Community-based Study of Needed US OIR Capabilities" and first authored chapter "The Co-evolution of Baryons, Black Holes, and Cosmic Structure"
\indent Spring 2011: Visited Washington DC as part of AAS Communicating with Washington project to promote astronomy to Kansas Delegation and NSF officials.\\
\indent Fall 2008, Fall 2009 -- Fall 2011, Fall 2012, NOAO TAC\\
\indent Aug. 2008-Nov. 2016: Gave 15 public talks in Lawrence area and was guest on 3 radio call-in shows in Kansas City (``The Walt Bodine Show'').\\
\indent Jan. 2006 - Aug. 2008: Lead scientist in {\it Spitzer
  Teachers} program to perform Spitzer research with nationally
selected high school teachers and students.  Co-recipient of NASA group acheivement award\\

\end{document}









      
                                    
