\subsection{co-PI Rudnick}
\label{Sec:prior_rudnick}
\subsubsection{Molecular Gas in Distant Galaxies'}

Co-PI Rudnick has recently completed his NSF project 1211358
``Characterizing the Molecular Gas Contents of High Redshift
Galaxies'' (\$306,754; 8/1/12-7/31/16).  

\textbf{Intellectual Merit:} This study was based on a large body of
JVLA data (200 hrs) on a $z=1.62$ galaxy cluster that was collected
between 2012 and October of 2014. The goal of this study was to
characterize the molecular gas content of high-$z$ galaxies by
observing CO. As a result of the studies of this cluster and of the CO
gas content of distant galaxies, Rudnick has authored or co-authored
seven papers since 2012 with a total of 185 citations
\citep{Papovich12,Rudnick12,Lotz13,Geach13,Wong14,Geach14,Tran15} as
well an ApJ paper that is in the resubmission process (Rudnick et
al.).  Since 2012, Rudnick has also given 28 oral presentations on
this NSF project.
%
Using the full JVLA data, Rudnick has securely detected CO(1--0) in
two massive and gas-rich galaxies in the $z=1.62$ cluster. These
galaxies have surprisingly low star formation efficiencies (SFE) for
their high mass and gas fraction \citep[e.g.][]{Genzel10}. This may
indicate the presence of environmental effects on the physical
conditions of the molecular gas and on the accretion of gas from the
cosmic web in a massive halo. These results appear in a paper that is
being resubmitted to ApJ after a favorable referee report.  The
expected publication date is early 2017.  As a direct result of this
project Rudnick has also organized a large consortium of scientists
who are seeking to use ALMA to make a census of the CO gas in distant
cluster galaxies.  They will resubmit a significantly sized proposal in
April 2017.

\textbf{Broader Impact:} Rudnick has completed the third year
(2013-2016) of an outreach program in close collaboration with Andrew
Bricker, a Physics teacher at Lawrence High School (LHS). Rudnick
developed and executed a year-long program in which the students
receive an introductory calculus-based astronomy course and perform a
bona fide research project. The goal of the class is to teach high
school students research methods, computing skills, the
electromagnetic spectrum, the nature of science, and science
communication while also giving the teacher new tools to teach
research-based activities in the classroom.  The project involves
using \textit{Spitzer}/MIPS 24\micron\ data to measure L$_{\rm IR}$
and SFRs for the galaxies in the infall regions of intermediate
redshift clusters from the ESO Distant Cluster Survey (EDisCS). The
students meet every day in a special class period. The teaching
assistant funded by the grant performed most of the instructional
duties and Rudnick attended class once a week.  30 students have gone
through the program during these three years.  This total was
comprised of $\sim 50\%$ underrepresented student groups: four African
American, four Latino, and 10 female students, two of which were also
women of color.  As described in \S\ref{BI} we employ extensive
assessment to understand our success at meeting learning goals.  This
program is continuing in 2016-2017 funded by another NSF project (see
\S\ref{Sec:gogreen_prior}).

\subsubsection{Galaxy Evolution in Distant Clusters}
\label{Sec:gogreen_prior}

co-PI Rudnick is in the beginning of his second year for the NSF
project 1517815, ``Collaborative Research: The GOGREEN Survey - Caring
About the Environment'' (\$347,556; 8/1/15-7/31/18).

\textbf{Intellectual Merit:} This study funds the US analysis efforts
for the international Gemini Observations of Galaxies in Rich Early
Environments (GOGREEN) project.  This project is based on the largest
Gemini Long and Large Program (PI: Michael Balogh), which is comprised
of 443 hours of Gemini imaging and spectroscopic observations
conducted over a 4 year period starting in Fall 2014.  The two main
components of this project are very deep Gemini optical spectroscopy
of a stellar mass ($M>10^{10}M_\odot$) limited sample of galaxies in
21 groups and clusters at $1<z<1.5$ and a large multiwavelength
imaging program. 

The goals of the project are to: \textbf{1)} Find the dominant modes
of satellite quenching at $z<1.5$; \textbf{2)} Determine how galaxies
populate dark matter halos as a function of environment at $z<1.5$;
\textbf{3)} Measure the relative timing of morphological
transformation and star-formation quenching; \textbf{4)} Constrain the
dominant driver of size growth in the early-type population at
\boldmath$z<1.5$.  

co-PI Rudnick is in charge of the imaging efforts for the whole
collaboration, which involves gathering
$UBVRIzYJK_S[3.6\micron][4.5\micron]$ data on a broad suite of
telescopes including Subaru, CFHT, Magellan, VLT, and \spitzer.  The
imaging of the southern clusters is 95\% complete and the northern
clusters only lack their NIR data.  We expect the imaging to be
completed by the Fall of 2017.

The strategy of the project is to obtain deep spectra over many
semesters on the faintest targets, and thus many of the science
publications will appear at the end of the proposal period.  However,
an initial data paper based on the first 30\% of the spectroscopic
data is in preparation with a Dec. 2016 submission target.

\textbf{Broader Impact:} Rudnick has extended his LHS outreach program
into the 2016-2017 academic year and will continue it through the
2017-2018 AY.  Changes that we have made this year include a much more
agressive targeting of URM students, which we have accomplished by
going to more junior students and not having as high of a math
prerequisite for entry into the program.  As a result we have our
highest fraction of URM students yet, with 1 native american, 2
african american, and 3 latino students, among which count two women
of color.  We are currently attempting to expand the program by using
seniors who have already completed the program as peer instructors.
This will allow us to grow without additional personnel costs.


\section{Broader Impacts}
\label{BI}

\subsection{University of Kansas}

\noindent \textbf{A High School Research Program:} Rudnick's broader
impact provides for the continuation of a successful research-based
outreach component at Lawrence High School (LHS; see
\S\ref{Sec:prior_rudnick}). This program is timely, as the Kansas
school system is an exemplar of the nationwide debate on the role of
science in the classroom. Crucial to increasing and clarifying the
role of science is educating students and training teachers.
%% What has become clear after the first year of this program is the
%% great value in bringing expert instructional support into the
%% classroom, as the partnering high school teacher lacks the research
%% experience at this point to adequately lead the program. The KU
%% employee (an ex-student) who is carrying out most of the in-class
%% instruction is proving crucial in this role. 
This program is currently funded through the end of the 2017-2018
academic year. The purpose of the current proposal is to: 1) continue
the funding of a KU student to work in the classroom, 2) develop the
project to the point of sustainability by training the high school
teacher in research-based teaching methods, 3) develop a peer
instruction model to expand the program to 20 students without
additional personnel costs, and 3) expand the program to 20
students. The school district fully supports this program (see letters
from McEwen and Bricker.)

\noindent \textbf{Implementation:} LHS has a 60\% higher percentage of
African Americans and a 340\% higher percentage of Native Americans,
when compared to other Kansas High Schools. Rudnick will continue his
successful recruitment of female students and those from
underrepresented minorities. The current proposal will extend the
ongoing project to measure SFRs of galaxies in the local filaments.
%% Coupled with student measurements
%% being made of EDisCS clusters at $0.4<z<0.8$, this will allow the
%% students to constrain the evolution in the SFRs of cluster galaxies
%% from $0.4<z<1.5$.
The requested funding will be used to hire a physics or astronomy
graduate student interested in education or an employee with similar
qualifications. This employee will be the main contact person in the
classroom and will lead the day-to-day instruction and supervision of
the learning teams during the execution of the project. Rudnick's
main tasks will be to coordinate the program, decide on the exact
curriculum based on our assessment process (see below), attend the
class once every week, and ensure that the program becomes sustainable
in future years. Through Rudnick's \textit{continuous} support and
heavy involvement in the program, the high school teacher is able to devote more of his time to training students to aid in peer instruction, which will allow us to expand in future years without additional personnel costs.

%% The KU graduate program has a solid record of students with an
%% education background or with an interest in education and outreach.
%% Rudnick currently has an ex-student employee who is excelling in his
%% role, but if needs be there will no problem recruiting a new student
%% with the necessary skills and motivation.

\noindent
\textbf{Assessment:} %% The program has been constructed according to the
%% theory of backwards design, in which the broad goals are defined
%% first, then specific goals, followed by the means of assessment, and
%% finally the details of implementation. The goals and implementation
%% are outlined in the previous paragraphs. 
The assessment consists of an end of project presentation and paper
for each student. Their presentations are made to KU faculty during a
mini-conference at KU. All students are given the pre- and post-course
Light and Spectra Concept Inventory \citep{Bardar06}.  We also make
students give multiple oral presentations throughout the semester and
have a rubric to evaluate their improvement over the course of the
project.
%% One ongoing task of the employee will be to research and develop
%% new assessment tools and to modify the program accordingly to the
%% results. As an example, in the Fall of 2014, Rudnick has decided
%% to focus more
%on student presentations in the astronomy instruction part of the
%course, to give the students practice in actively seeking out and
%presenting their own astronomy knowledge.
This project satisfies important elements of several of the Kansas
state science standards, i.e.~``Science as Inquiry'' via the process
of research and of communicating their results, benchmark 2 and 3 of
the Physics standards via learning about the electromagnetic spectrum
and how it relates to astronomical phenomena, benchmark 4 of the earth
and space sciences standards relating to general astronomy, and
benchmark 2 of the ``history and nature of science'' via the
understanding gained of the scientific process.
