\subsection{co-PI Rudnick}
\label{Sec:prior_rudnick}
\subsubsection{Molecular Gas in Distant Galaxies'}

Co-PI Rudnick has recently completed his NSF project 1211358
``Characterizing the Molecular Gas Contents of High Redshift
Galaxies'' (\$306,754; 8/1/12-7/31/16).  

\textbf{Intellectual Merit:} This study was based on a large body of
JVLA data (200 hrs) on a $z=1.62$ galaxy cluster that was collected
between 2012 and October of 2014. The goal of this study was to
characterize the molecular gas content of high-$z$ galaxies by
observing CO. As a result of the studies of this cluster and of the CO
gas content of distant galaxies, Rudnick has authored or co-authored
seven papers since 2012 with a total of 185 citations
\citep{papovich12,rudnick12,lotz13,geach13,wong14,geach14,tran15} as
well an ApJ paper that is in the resubmission process (Rudnick et
al.).  Since 2012, Rudnick has also given 28 oral presentations on
this NSF project.
%
Using the full JVLA data, Rudnick has securely detected CO(1--0) in
two massive and gas-rich galaxies in the $z=1.62$ cluster. These
galaxies have surprisingly low star formation efficiencies (SFE) for
their high mass and gas fraction \citep[e.g.][]{genzel10}. This may
indicate the presence of environmental effects on the physical
conditions of the molecular gas and on the accretion of gas from the
cosmic web in a massive halo. These results appear in a paper that is
being resubmitted to ApJ after a favorable referee report.  The
expected publication date is early 2017.  As a direct result of this
project Rudnick has also organized a large consortium of scientists
who are seeking to use ALMA to make a census of the CO gas in distant
cluster galaxies.  They will resubmit a significantly sized proposal in
April 2017.

\textbf{Broader Impact:} Rudnick has completed the third year
(2013-2016) of an outreach program in close collaboration with Andrew
Bricker, a Physics teacher at Lawrence High School (LHS). Rudnick
developed and executed a year-long program in which the students
receive an introductory calculus-based astronomy course and perform a
bona fide research project. The goal of the class is to teach high
school students research methods, computing skills, the
electromagnetic spectrum, the nature of science, and science
communication while also giving the teacher new tools to teach
research-based activities in the classroom.  The project involves
using \textit{Spitzer}/MIPS 24\micron\ data to measure L$_{\rm IR}$
and SFRs for the galaxies in the infall regions of intermediate
redshift clusters from the ESO Distant Cluster Survey (EDisCS). The
students meet every day in a special class period. The teaching
assistant funded by the grant performed most of the instructional
duties and Rudnick attended class once a week.  30 students have gone
through the program during these three years.  This total was
comprised of $\sim 50\%$ underrepresented student groups: four African
American, three Hispanic, one Native American, and 10 female students,
two of which were also women of color.  As described in \S\ref{BI} we
employ extensive assessment to understand our success at meeting
learning goals.  This program is continuing in 2016-2017 funded by
another NSF project (see \S\ref{Sec:gogreen_prior}).

\subsubsection{Galaxy Evolution in Distant Clusters}
\label{Sec:gogreen_prior}

co-PI Rudnick is in the beginning of his second year for the NSF
project 1517815, ``Collaborative Research: The GOGREEN Survey - Caring
About the Environment'' (\$347,556; 8/1/15-7/31/18).

\textbf{Intellectual Merit:} This study funds the US analysis efforts
for the international Gemini Observations of Galaxies in Rich Early
Environments (GOGREEN) project.  This project is based on the largest
Gemini Long and Large Program (PI: Michael Balogh), which is comprised
of 443 hours of Gemini imaging and spectroscopic observations
conducted over a 4 year period starting in Fall 2014.  The two main
components of this project are very deep Gemini optical spectroscopy
of a stellar mass ($M>10^{10}M_\odot$) limited sample of galaxies in
21 groups and clusters at $1<z<1.5$ and a large multiwavelength
imaging program. 

The goals of the project are to: \textbf{1)} Find the dominant modes
of satellite quenching at $z<1.5$; \textbf{2)} Determine how galaxies
populate dark matter halos as a function of environment at $z<1.5$;
\textbf{3)} Measure the relative timing of morphological
transformation and star-formation quenching; \textbf{4)} Constrain the
dominant driver of size growth in the early-type population at
\boldmath$z<1.5$.  

co-PI Rudnick is in charge of the imaging efforts for the whole
collaboration, which involves gathering
$UBVRIzYJK_S[3.6\micron][4.5\micron]$ data on a broad suite of
telescopes including Subaru, CFHT, Magellan, VLT, and $Spitzer$.  The
imaging of the southern clusters is 95\% complete and the northern
clusters only lack their NIR data.  We expect the imaging to be
completed by the Fall of 2017.

The strategy of the project is to obtain deep spectra over many
semesters on the faintest targets, and thus many of the science
publications will appear at the end of the proposal period.  However,
an initial data paper based on the first 30\% of the spectroscopic
data is in preparation with a Dec. 2016 submission target.

\textbf{Broader Impact:} Rudnick has extended his LHS outreach program
into the 2016-2017 academic year and will continue it through the
2017-2018 AY.  Changes that we have made this year include a much more
agressive targeting of URM students, which we have accomplished by
going to more junior students and not having as high of a math
prerequisite for entry into the program.  As a result we have our
highest fraction of URM students yet, with one Native American, three
African American, two Hispanic students and three women ,one of which
is a women of color.  We are currently attempting to expand the
program by using seniors who have already completed the program as
peer instructors.  This will allow us to grow without additional
personnel costs.


w