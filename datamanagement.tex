\documentclass[11pt,preprint]{aastex}

\usepackage{graphicx}
\usepackage[vmargin=1in,hmargin=1in]{geometry}
\usepackage{natbib}
\usepackage{epsfig}
% \usepackage{epsfig, xcolor, footnote, mdwlist, wrapfig}
% \usepackage{hyperref}
% \usepackage[flushleft]{threeparttable}
% \usepackage{sidecap}
% \usepackage{simplemargins}
% \usepackage[compact]{titlesec}
% \usepackage{times}
% \usepackage{amsmath}
% \usepackage[]{natbib}
% \usepackage{sidecap}

%\usepackage[font=footnotesize]{caption}
%\usepackage{caption}
%\usepackage[font=footnotesize, figurename=Figure, labelsep=endash]{caption}
%\usepackage{multirow}

\setlength\parindent{0pt}


%\setlength{\oddsidemargin}{-0.2 in}
%\setlength{\evensidemargin}{-0.2 in}

\pagenumbering{arabic} %roman/Roman/arabic
\pagestyle{plain}

\renewcommand{\baselinestretch}{0.94} 

\begin{document}


{\Large{\bf Data Management Plan}}
%\vspace*{2mm}


Expected Data and Usage:  All software algorithms, programs for data analysis and for simulations will be stored. The data collected, processed, and analyzed will be used to generate peer-reviewed research publications as well as local, national, and international research presentations. 

Data Storage and Preservation: The PI will be responsible for storing data and making decisions regarding overall data management. A master copy of all data, program, algorithms, simulations, and analyses will be kept, as well as a copy on physical media (DVD-R, CD-R, memory card, or external USB hard drive) for archival purposes that will be stored outside the laboratory. As such, these data files are available at all times on the laboratory computers, but are not susceptible to loss due to hard drive failure or laboratory damage. In addition, all desktops at Siena's computational laboratories and used in this project will be equipped with RAID1 dual hard-drive technology, which provides fault tolerance from disk errors or failures. 

Data Sharing and Dissemination: 
We will release all of our data with our final publication.  This will include 
reduced, calibrated, and astrometrically aligned HI,
Halpha, and SDSS maps with integrated CO fluxes.
We will produce a variety of measurements of the gas content of
filament galaxies.  We will produce a catalog of CO, HI, dust, Halpha
properties to be released with our final paper.
While this data set of derived information will not be exceptionally large in volume, it will be of value to other observational astronomers and theorists working on galaxy evolution. To facilitate the widespread use of these data, we will publish the full catalog of measured and derived quantities on the web and in the on-line version of Astrophysical Journal Supplement. In addition, we will work with the NASA/IPAC Extragalactic Database (NED) to ensure that our data are discoverable to a larger user base. Co-I Vandana Desai, resident at IPAC, will act as our NED liason to ensure that this gets done in a timely manner.

\end{document}