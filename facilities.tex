%%%%%%%%%%%%%%%%%%%%%%%%%%%%%%%%%%%%%%%%%%%%%%%%%%%%%%%%%%%%%%%%%%%%%%%%%%%%%%%%
\documentclass[11pt]{article}
\pagenumbering{arabic}
%%%%%%%%%%%%%%%%%%%%%%%%%%%%%%%%%%%%%%%%%%%%%%%%%%%%%%%%%%%%%%%%%%%%%%%%%%%%%%%%

%%%%%%%%%%%%%%%%%%%%%%%%%%%%%%%%%%%%%%%%%%%%%%%%%%%%%%%%%%%%%%%%%%%%%%%%%%%%%%%%
\usepackage{amsmath}
\usepackage[pdftex]{graphicx}
\usepackage{color}
\usepackage{subfigure}
\usepackage{enumitem}

%%%%%%%%%%%%%%%%%%%%%%%%%%%%%%%%%%%%%%%%%%%%%%%%%%%%%%%%%%%%%%%%%%%%%%%%%%%%%%%%
\hsize=6.5in
\hoffset=-0.75in
\setlength{\textwidth}{6.5in}
\setlength{\textheight}{9in}
\setlength{\voffset}{0pt}
\setlength{\topmargin}{0pt}
\setlength{\headheight}{0pt}
\setlength{\headsep}{0pt}

\vsize=9.0in
\renewcommand\baselinestretch{0.93}
%%%%%%%%%%%%%%%%%%%%%%%%%%%%%%%%%%%%%%%%%%%%%%%%%%%%%%%%%%%%%%%%%%%%%%%%%%%%%%%%

\normalsize

%%%%%%%%%%%%%%%%%%%%%%%%%%%%%%%%%%%%%%%%%%%%%%%%%%%%%%%%%%%%%%%%%%%%%%%%%%%%%%%
\usepackage{hyperref}
\hypersetup{
    colorlinks=true,   % false: boxed links; true: colored links
    linkcolor=red,     % color of internal links
    citecolor=red,     % color of links to bibliography
    filecolor=magenta, % color of file links
    urlcolor=blue      % color of external links
}
%%%%%%%%%%%%%%%%%%%%%%%%%%%%%%%%%%%%%%%%%%%%%%%%%%%%%%%%%%%%%%%%%%%%%%%%%%%%%%%

%%%%%%%%%%%%%%%%%%%%%%%%%%%%%%%%%%%%%%%%%%%%%%%%%%%%%%%%%%%%%%%%%%%%%%%%%%%%%%%%
%%%%%%%%%%%%%%%%%%%%%%%%%%%%%%%%%%%%%%%%%%%%%%%%%%%%%%%%%%%%%%%%%%%%%%%%%%%%%%%%
\begin{document}

%%%%%%%%%%%%%%%%%%%%%%%%%%%%%%%%%%%%%%%%%%%%%%%%%%%
\centerline{\Large\bf Facilities, equipment, and other resources}
\vspace{0.2cm}
%%%%%%%%%%%%%%%%%%%%%%%%%%%%%%%%%%%%%%%%%%%%%%%%%%%

Siena College provides a unique array of outstanding scientific, reference and research facilities 
for a small liberal arts college.  The J. Spencer and Patricia Standish Library is a 72,000 square foot 
building that provides wireless access, over 100 computer workstations, a computer lab and a 40-seat 
screening room.  Moreover, through collaborative exchange agreements with nearby Rensselaer Polytechnic Institute (RPI) 
and the University of Albany, there is extensive access to first-rate research library facilities.

The Morrell Science Center (MSC) is a 55,000 square foot science center with 24 research labs, 10 teaching labs and 
three support areas on three floors. There is also a small machine shop for manufacture of prototype parts and 
lab apparatus.  The School of Science maintains an independent network that includes two Apache web servers integrated 
with two file servers (Dell PowerEdge 2650+ servers with Dual Xeon processors running RedHat Enterprise Linux ES 3.0), 
and an Oracle-based database server.  Collectively, this amounts to over 10TB of total storage. The school also maintains 
twelve electronically enhanced classrooms (EECs) and labs that each includes a projector, sound system, and podium computer. 
The Department of Physics uses technology extensively for teaching, with both Python and MATLAB as the standard program in the 
majors sequence.  LabView and Mathematica are also used. 

Siena has recently (Summer 2014) completed the construction of the 
Stewarts Advanced Instrumentation \& Technology (SAInT) Center with the goal of 
establishing Siena as a leader in undergraduate education in scientific 
instrumental resources and training. The Center is located in a newly renovated 
laboratory space in MSC. New Instrumentation was 
purchased including multiple mass spectrometers, 
an atomic force microscope, a scanning electron microscope, and other
analytic equipment.
The college has hired Dr. Kristopher Kolonko to act as the director of the SAInT Center to 
maintain the instrumentation, provide training and support for students and faculty, and 
coordinate of the use of instrumentation with both internal and external users.

Another new addition (2013) is the Siena College High Performance Computing Center (HPCC). 
The HPCC cluster has 252 2.3 GHz Intel Xeon (E5-2630) 
cores and 20.5 TB of global storage.  Each worker node has 500 GB of local storage and 32 GB of 
RAM. A full suite of software tools and compilers are available on the cluster and 
both the PI and students have access to this resource.


Over the past decade, Siena has undergone a transformation as it relates to sponsored research activities and 
undergraduate research participation due to the exponential growth in grant monies procured from federal funding agencies 
such as NSF, NASA, NEH and the U.S. Department of Education.  The College established a Center for 
Undergraduate Research and Creative Activities (CURCA) to foster a campus-wide culture in which all undergraduates are engaged in 
investigations conducted in collaboration with a faculty mentor that makes an original intellectual or creative contribution 
to a discipline or the community. The center's resources facilitate and enhance the research experience for all undergraduates, 
including students funding by external grants. Students are encouraged to take part in national poster events such as the 
Council on Undergraduate Research ``Posters on the Hill" event held each year at the Capitol in Washington, DC, and the 
National Conference for Undergraduate Research event that is attended by more than 200 students and faculty from some of 
the best colleges and universities in the nation. The ultimate goal of the center is to prepare highly engaged and 
motivated students to pursue post-baccalaureate opportunities upon graduation at top-tier national universities. 
CURCA has been particularly supportive of the work by the PI on CMS efforts, supplementing 
the previous grant with travel funds so that additional students could present their work at 
American Physical Society meetings.

This proposal is structured around a collaboration with Cornell University that will 
provide the PI and Siena students with access to Cornell's graduate students,
post-docs, and faculty. Siena collaborators have access to a subset of
Cornell's computers in the High Energy Physics experimental group and are
able to access these computers remotely from Siena.
Cornell University is a three-hour drive from Siena College.

\end{document}
%%%%%%%%%%%%%%%%%%%%%%%%%%%%%%%%%%%%%%%%%%%%%%%%%%%%%%%%%%%%%%%%%%%%%%%%%%%%%%%
%%%%%%%%%%%%%%%%%%%%%%%%%%%%%%%%%%%%%%%%%%%%%%%%%%%%%%%%%%%%%%%%%%%%%%%%%%%%%%%
