%%%%%%%%%%%%%%%%%%%%%%%%%%%%%%%%%%%%%%%%%%%%%%%%%%%%%%%%%%%%%%%%%%%%%%%%%%%%%%%%
\documentclass[11pt,preprint]{aastex}

\usepackage{graphicx}
\usepackage[vmargin=1in,hmargin=1in]{geometry}
\usepackage{natbib}
\usepackage{epsfig}
\setlength{\parindent}{0in}

\begin{document}

%%%%%%%%%%%%%%%%%%%%%%%%%%%%%%%%%%%%%%%%%%%%%%%%%%%

\noindent
{\Large\bf Facilities, equipment, and other resources}


Siena College provides a unique array of outstanding scientific, reference and research facilities 
for a small liberal arts college.  The J. Spencer and Patricia Standish Library is a 72,000 square foot 
building that provides wireless access, over 100 computer workstations, a computer lab and a 40-seat 
screening room.  Moreover, through collaborative exchange agreements with nearby Rensselaer Polytechnic Institute (RPI) 
and the University of Albany, there is extensive access to first-rate research library facilities.

The Morrell Science Center (MSC) is a 55,000 square foot science center with 24 research labs, 10 teaching labs and 
three support areas on three floors. There is also a small machine shop for manufacture of prototype parts and 
lab apparatus.  The School of Science maintains an independent network that includes two Apache web servers integrated 
with two file servers (Dell PowerEdge 2650+ servers with Dual Xeon processors running RedHat Enterprise Linux ES 3.0), 
and an Oracle-based database server.  Collectively, this amounts to over 10TB of total storage. The school also maintains 
twelve electronically enhanced classrooms (EECs) and labs that each includes a projector, sound system, and podium computer. 
The Department of Physics uses technology extensively for teaching, with both Python and MATLAB as the standard program in the 
majors sequence.  LabView and Mathematica are also used. 

Siena has recently (Summer 2014) completed the construction of the 
Stewarts Advanced Instrumentation \& Technology (SAInT) Center with the goal of 
establishing Siena as a leader in undergraduate education in scientific 
instrumental resources and training. The Center is located in a newly renovated 
laboratory space in MSC. New Instrumentation was 
purchased including multiple mass spectrometers, 
an atomic force microscope, a scanning electron microscope, and other
analytic equipment.
The college has hired Dr. Kristopher Kolonko to act as the director of the SAInT Center to 
maintain the instrumentation, provide training and support for students and faculty, and 
coordinate of the use of instrumentation with both internal and external users.

Another new addition (2013) is the Siena College High Performance Computing Center (HPCC). 
The HPCC cluster has 252 2.3 GHz Intel Xeon (E5-2630) 
cores and 20.5 TB of global storage.  Each worker node has 500 GB of local storage and 32 GB of 
RAM. A full suite of software tools and compilers are available on the
cluster.  co-I Vernizzi will use this cluster
to fit 2-dimensional GALFIT models to the WISE images by
stochastically sampling the range of potential input parameters.

Siena College provides all the structural scientific and research facilities
necessary to carry out the proposed research program.  The PI has dedicated
office space in Roger Bacon Hall, which houses the Department of Physics \&
Astronomy within the School of Science.  She and her students utilize a
$600$~square foot laboratory dedicated for astrophysics research.  This lab
currently holds four dedicated, reasonably modern imac computers and a
linux server, providing remote access to a departmental server housed in the College's
Data Center.  





\end{document}
%%%%%%%%%%%%%%%%%%%%%%%%%%%%%%%%%%%%%%%%%%%%%%%%%%%%%%%%%%%%%%%%%%%%%%%%%%%%%%%
%%%%%%%%%%%%%%%%%%%%%%%%%%%%%%%%%%%%%%%%%%%%%%%%%%%%%%%%%%%%%%%%%%%%%%%%%%%%%%%
