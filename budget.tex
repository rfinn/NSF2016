\documentclass[preprint,11pt]{aastex}
\usepackage{graphicx}
\usepackage[vmargin=1in,hmargin=1in]{geometry}
\usepackage{natbib}
\usepackage{epsfig}
\setlength{\parindent}{0in}



\newcommand{\ha}{H$\alpha$}
\begin{document}

\begin{center}
{\bf \large A. Salaries and Wages}
\end{center}

The normal course load at Siena College is 4 classes per semester.  Siena
has agreed to a 25\% course reduction to increase the amount of time
that research-active faculty can allot to research.  We have requested additional funds to reduce
Finn's teaching load by one course during the spring semester for the
first two years so that she can
focus half of her time on research.  The \ha \ imaging campaigns will
occur during the spring semester, and the course release will provide valuable time
for the PI to develop the \ha \ reduction pipeline and observe, reduce and analyze
the \ha \ imaging data.  In addition, she will prepare the summer research
projects for participating undergraduates.  

We request 2 months summer salary each year of the grant for PI Finn
to oversee the reduction and analysis of the \ha \ and WISE images,
and to supervise the ten-week research experience four undergraduate students.  
Salaries are assumed to increase by 3\% per year.

\begin{center}
{\bf \large B. Other Personnel}
\end{center}

We have included one month of summer salary for Graziano Vernizzi
during the first year of this grant.  During this time, he will
develop the code to stochastically sample parameter space when fitting
two-dimensional models to the WISE images that will run at Siena's
High Performance Computing Center.

The PI will employ 4 undergraduates during summer months
for the 3 years covered by this proposal.  The current rate is \$10.50
per hour, and the undergraduate salaries include 35 hours per week 
for 8 weeks during the summer.  
We have also included funds to cover the cost of housing for the
summer students (\$135/week).  This helps defray the expense of
participating in summer research.

We have included funds for 2 students to continue work during the
academic year (10 hrs per week).

\begin{center}
{\bf \large C. Fringe Benefits}
\end{center}

The fringe rate for full-time faculty 
is 42.2\% for the academic and and 8.6\% for summer salary.

The fringe rate for undergraduate students is 
5.1\% for summer research and for work completed during the
academic year.

\begin{center}
{\bf \large D. Equipment}
\end{center}

We have requested funding \$5000 the first year to 
to purchase two computers, one for the PI and the second to be used by
undergraduates.  We request an additional \$2000 to purchase external
hard drives to back up all imaging data.

\begin{center}
{\bf \large E. Travel}
\end{center}

The PI and
two undergraduate student will attend 
the winter meeting of the American Astronomical
Society during the second and third years to present results from this survey.  
The allowance per participant 
includes: \$500 for airfare; registration fee (\$500 for PI, 
\$200 for undergraduates); \$250 for hotel times 5 nights
(double occupancy for undergraduates); \$60 per day allowance for food.  

This proposal will fund travel for the PI, one co-I and two undergraduate students 
to Kitt Peak National Observatory for one observing trip each year, with an allowance 
per participant per trip of: \$500 for airfare;
\$125 per night for room and board at the observatory.
We include an additional \$200 for a rental car or 
alternate transporation to and from the airport.
If we have perfect weather during the first two years, we won't need
to observe during the third year.  However, 100\% clear weather is not
realistic, and technical issues sometimes arise at the telescope.  We
therefore plan for an additional observing trip during the third year.


This proposal will fund travel for the PI and co-I Desai to an international team
meeting during the second year with an allowance of:
\$1400 for airfare; \$160 for hotel times 7 nights;
\$75 per day allowance for food.

We will hold a team meeting at Siena College during the first and
third years of the grant.  
The proposal will fund travel for co-I Desai including
\$500 for airfare, \$150 per night for 4 nights for a hotel,
and \$60 per diem.  

\begin{center}
{\bf \large F. Participant Support Costs}
\end{center}

\begin{center}
{\bf \large G. Other Direct Costs}
\end{center}


%Materials/Supplies

Funds for publication are requested at the {\it Astrophysical Journal} rate of approximately \$150 per page.
We expect a total of six publications to result from this proposal.
In his collaborative proposal, co-PI Greg Rudnick has budgeted for the
first three papers, and here we budget for the last three papers which
we expect to publish in the third year of the grant.  We estimate 10
page papers (\$125 per page), for a total publication cost of \$3750 per paper.


%Funds are requested for an educational consultant of  
%\$5000 for the first year for the consultant to design
%and implement as assessment plan and \$3000 each 
%subsequent year to perform a 
%year-end evaluation.

%Darren Broder (CV attached as Supplemental 
%Document) will organize and run the fall workshop for high school physics teachers
%for a fixed fee of \$1000 per year, and we have requested funds to cover
%this.




\begin{center}
{\bf \large G. Indirect Costs}
\end{center}

The indirect cost rate at Siena College is 21.3\%, excluding
equipment, Participant Support, and Subawards.  

\end{document}